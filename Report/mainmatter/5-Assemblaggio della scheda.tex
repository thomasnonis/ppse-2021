\chapter{Assemblaggio della scheda}

Come già accennato in precedenza, una volta finita la fase di progetto della scheda, abbiamo inviato al Professore tutti i file relativi alla produzione (\textit{Gerber}, \textit{Drill} e \textit{BOM}), il quale ha poi provveduto ad inviare le PCB in produzione, ordinando anche uno stencil, e ad effettuare gli ordini per i componenti necessari.
Dopo circa due settimane dall’invio eravamo finalmente in possesso delle schede e della maggior parte dei componenti necessari, quindi abbiamo potuto cominciare l’assemblaggio, che è stato svolto sotto la supervisione del Professore presso il laboratorio di elettronica del FabLab. \\\
Prima di iniziare con la fase di assemblaggio, abbiamo effettuato un’ispezione visiva delle PCB sotto una lente per verificarne la correttezza; grazie a questa abbiamo identificato un solo problema: il marker sul layer \textit{silkscreen} per il pin 1 del microcontrollore non è stato stampato, probabilmente perché era troppo piccolo. \\
Abbiamo poi ispezionato anche lo \textit{stencil} per la pasta saldante ed abbiamo notato che il produttore non ha usato come riferimento il layer apposito, ma ha usato quello per la \textit{soldermask}. Per questo motivo sullo stencil erano presenti aperture anche in corrispondenza dei pad \textit{through-hole} del \textit{GPS} (che abbiamo provveduto a coprire con del nastro \textit{kapton}) e le aperture dei fiducials sono risultate più grandi del previsto, anche se questo non ha alcuna importanza per la realizzazione, quindi siamo passati allo step successivo. \\
Per iniziare l’assemblaggio abbiamo fissato una PCB con altre 4 utilizzando del nastro ed abbiamo allineato lo stencil fissandolo da un lato con altro nastro per poterlo alzare una volta finito, come in figura.

\begin{center}
    \includegraphics[width=9cm]{figures/image106.jpg}
    \captionsetup{type=figure}
    \captionof{figure}{Il sistema utilizzato per fissare la scheda e lo stencil}
\end{center}

\noindent Abbiamo quindi depositato, con l’aiuto di una spatola, la pasta saldante, controllando di riempire ogni apertura con una quantità adeguata. Verificata la corretta applicazione abbiamo tolto la scheda dal supporto da noi creato ed abbiamo cominciato ad applicare i componenti \textit{SMD} con l’aiuto di pinzette di precisione e controllando continuamente il layout.

\begin{center}
    \includegraphics[width=0.6\textwidth]{figures/image107.jpg}
    \captionsetup{type=figure}
    \captionof{figure}{PCB con pasta applicata ed i primi componenti posizionati}
\end{center}

\noindent Per tenere traccia del progresso abbiamo usato estensivamente un utilissimo plugin per KiCad chiamato \textit{iBOM}, che permette di visualizzare il layout del PCB, associando ad ogni componente la sua posizione in una grafica molto intuitiva, e di contrassegnare ogni componente come piazzato.

\begin{center}
    \includegraphics[width=\textwidth]{figures/image108.png}
    \captionsetup{type=figure}
    \captionof{figure}{Interfaccia del plugin iBOM}
\end{center}

\noindent Alcuni dei componenti utilizzati presentavano un modello diverso da quello previsto in fase di progettazione a seconda delle disponibilità\textit{ in-house }del professore ma, essendo una scheda di prototipazione e non una scheda di produzione, sono risultati ugualmente adatti, senza alterarne il funzionamento. Infine, abbiamo posto la scheda su di una piastra elettrica riscaldante controllata per fare il reflow della pasta e saldare definitivamente i componenti \textit{SMD}. A questo punto abbiamo saldato manualmente, con un classico saldatore a stilo, i restanti componenti \textit{THT}, ovvero buzzer, pulsanti, portafusibile e connettori.

\noindent Abbiamo ripetuto questi passaggi per sei volte, realizzando così sei PCBA, uno per ogni membro del gruppo più uno per il professore.\\
Finito l’assemblaggio, ci siamo accorti che per una particolare scheda i pin del microcontrollore non sembravano 
essere fissati a dovere ed infatti testando la sua alimentazione abbiamo potuto constatare il surriscaldamento eccessivo 
del componente. Perciò per rimediare all’errore il microcontrollore è stato dissaldato, ma nel farlo, le tre piste che collegavano 
i rispettivi pin si sono alzate, non consentendo così l’utilizzo della scheda. Un altro errore è sorto durante la fase di testing, 
infatti, misurando il segnale di comando in input ai motori, ci siamo accorti che al transistor a canale N montato sulla scheda era 
stata assegnata la piedinatura scorretta. Pertanto, abbiamo dovuto dissaldare i transistor atti al pilotaggio dei motori per poi 
saldarli nuovamente nel verso corretto.\\
Come già anticipato non abbiamo potuto montare il led RGB a causa dell'errore nell'assegnamento del pinout e dell'errore nello schematico.