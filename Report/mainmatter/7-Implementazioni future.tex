\chapter{Implementazioni future}

Considerando i componenti mancanti, le funzionalità che vorremmo
implementare in un prossimo futuro sarebbero le seguenti:

\begin{itemize}
\item
  \begin{quote}
  Interfaccia utente attraverso il display OLED
  \end{quote}
\item
  \begin{quote}
  Sensing di tensione e corrente prodotti dal pannello
  \end{quote}
\item
  \begin{quote}
  Struttura per pannello, motori e sensori
  \end{quote}
\item
  \begin{quote}
  Interfaccia Python via COM port con visualizzazione valori,
  statistiche e controllo manuale motori
  \end{quote}
\end{itemize}

L'interfaccia utente nel display OLED mostrerà: tensione pannello e
corrente pannello, posizione del sole e posizione pannello (angolo di
Azimuth e angolo di Elevazione), orario e posizione geografica.

\begin{center}
\includegraphics[width=4.68229in,height=3.4426in]{figures/image27.png}
\captionsetup{type=figure}
\captionof{figure}{Esempio di interfaccia utente su display OLED}
\end{center}

La struttura di supporto per il pannello servirà prima di tutto a
consentire al pannello di muoversi nei due gradi di libertà (rotazione
intorno all'asse Z e rotazione intorno all'asse X) e poi per contenere
magnetometro ed accelerometro in modo che questi rilevino esattamente la
posizione del pannello (dovranno quindi essere fisicamente attaccati al
pannello e dovranno muoversi con esso).

\begin{center}
\includegraphics[width=3.63in,height=2.72in]{figures/image52.png}
\includegraphics[width=3.63in,height=2.72in]{figures/image68.png}
\includegraphics[width=3.63in,height=2.72in]{figures/image20.png}
\captionsetup{type=figure}
\captionof{figure}{Esempio di struttura di supporto per pannello e motori di tipo servo\\
(Credits:\href{https://www.thingiverse.com/thing:53321}{\underline{https://www.thingiverse.com/thing:53321}})}
\end{center}

Per quanto riguarda l'interfaccia grafica per la connessione seriale via
USB, si pensava di sviluppare in GUI (Graphical User Interface) in
linguaggio Python che consentisse di controllare manualmente le
posizioni dei motori, di abilitare il tracking o il return to home e
che, attraverso l'uso di piccoli grafici, mostrasse delle semplici
statistiche sulla potenza e l'energia prodotte dal pannello
fotovoltaico.

\begin{center}
\includegraphics[width=4.68229in,height=3.4426in]{figures/image47.png}
\captionsetup{type=figure}
\captionof{figure}{Esempio di GUI sviluppata in python}
\end{center}

